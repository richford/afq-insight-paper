\section*{Abstract}

The white matter contains long-range connections between different
brain regions and the organization of these connections holds important
implications for brain function in health and disease. Tractometry
uses diffusion-weighted magnetic resonance imaging (dMRI) data
to quantify tissue properties (e.g. fractional anisotropy (FA),
mean diffusivity (MD), etc.), along the trajectories of these
connections\cite{yeatman2012tract}. Statistical inference from
tractometry usually either (a) averages these quantities along
the length of each bundle in each individual, or (b) performs analysis
point-by-point along each bundle, with group comparisons or regression
models computed separately for each point along every one of the bundles.
These approaches are limited in their sensitivity, in the former case, or
in their statistical power, in the latter.
In the present work, we developed a method based on the sparse group
lasso (SGL) \cite{simon2013sparse} that takes into account tissue
properties measured along all of the bundles, and selects informative
features by enforcing sparsity, not only at the level of individual
bundles, but also across the entire set of bundles and all of the measured
tissue properties. The sparsity penalties for each of these constraints
is identified using a nested cross-validation scheme that guards
against over-fitting and simultaneously identifies the correct
level of sparsity. We demonstrate the accuracy of the method in two
settings: i) In a classification setting, patients with amyotrophic
lateral sclerosis (ALS) are accurately distinguished from matched
controls \cite{sarica2017corticospinal}. Furthermore, SGL automatically
identifies FA in the corticospinal tract as important for this
classification -- correctly finding the parts of the white matter known
to be affected by the disease. ii) In a regression setting, dMRI is
used to accurately predict ``brain age'' \cite{yeatman2014lifespan,
Brown2012-so}. In this case, the weights are distributed throughout the
white matter indicating that many different regions of the white matter
change with development and contribute to the prediction of age. Thus,
SGL makes it possible to leverage the multivariate relationship between
diffusion properties measured along multiple bundles to make accurate
predictions of subject characteristics while simultaneously discovering
the most relevant features of the white matter for the characteristic of
interest.
