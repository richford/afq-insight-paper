\section*{Abstract}

The white matter contains long-range connections between different brain
regions. Tractometry uses diffusion-weighted magnetic resonance imaging (dMRI)
data to quantify tissue properties, such as fractional anisotropy (FA) and mean
diffusivity (MD), along the trajectories of these connections in the human brain
\emph{in vivo}\cite{yeatman2012tract}. Statistical inference from tractometry
data usually either averages these quantities along the length of the tract in
each individual, or performs analysis point-by-point along each tract, with
group comparisons or regression models computed separately for each point along
every one of the tracts. Alternatively, tissue properties are computed for a
specific tract of interest based on an a priori hypothesis. In the present work,
we developed a method based on the sparse group lasso (SGL)
\cite{simon2013sparse} that takes into account tissue properties measured along
the tracts, and selects informative features by enforcing sparsity, both at the
level of individual tracts and tissue properties, but also across the entire set
of tracts and all of the measured tissue properties. The sparsity penalties for
each of these constraints is identified using a nested cross-validation scheme
that guards against over-fitting and simultaneously identifies the correct level
of sparsity, both at the group level and overall. We demonstrate the accuracy of
the method in two settings: In a regression setting, dMRI is used to accurately
predict ``brain age'' \cite{yeatman2014lifespan, Brown2012-so}. In a
classification setting, patients with amyotrophic lateral sclerosis (ALS) are
accurately distinguished from matched controls\cite{sarica2017corticospinal}. In
addition, SGL automatically identifies FA in the corticospinal tract as
important for this classification -- correctly finding the parts of the white
matter known to be affected by ALS. \ariel{Still need a punchy ending sentence
here}