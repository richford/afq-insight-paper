\section*{Results and discussion}

\subsection*{SGL of tractometry data in a regression setting}

To test the performance of SGL in tractometry data in a continuous regression
task, we focus here on the prediction of biological age from the tractometry
data. Prediction of ``brain age'' is a commonly undertaken task. This is both
because it operates on a natural scale, with meaningful and easily understood
units, as well as because predictions of brain age, and deviations from accurate
prediction are diagnostic of overall brain age \ariel{citation needed}. The WH
dataset used here contains data from 76 healthy subjects, ranging between 6
years and 50 years of age \cite{yeatman2014lifespan}. In this case, biological
age was used as the predicted variable ($y$ in equation \ref{eq:lm}). SGL was
fit to tractometry-extracted features: FA and MD in 20 major brain tracts, with
each tract divided into 100 nodes. To evaluate the fit of the model, we used a
nested cross-validation procedure. In this procedure, batches of subjects are
held out. For each batch (or fold), the model is fully fit without this data.
Then, once the parameters are fixed, the model is inverted to predict the ages
of held out subjects based on the linear coeffiecients and the static
non-linearity. data is held out, and a model age was predicted using a model
that was fit without that subject. This scheme automatically finds the right
level of regularization and fits the coefficients to the ill-posed linear model,
while guarding against overfitting. SGL accurately predicts the age of the
subjects in this procedure, with a mean absolute error of 3.6 years (Figure
\ref{fig:regress-results}, left panel). This is lower than the results of a
recent study that predicted age in a large sample, based on diffusion MRI
features \cite{Richard2018-ux}. Nevertheless, older subjects have higher
residual variance, reflecting the automatically-chosen log-transformation and
implying that brain age becomes more difficult to predict as we age
chronologically (\ref{fig:regress-result}, right panel). The model weights are
distributed over many different tracts and dMRI tissue properties (Figure
\ref{fig:regress-beta} left). This demonstrates that SGL is not coerced to
produce overly sparse results when a more accurate model requires a dense
selection of features. Furthermore, looking closer at a selection of tracts
where high coefficients are found demonstrates that diffusion properties (FA, in
this case) are different in different age groups in parts of the tracts where
these higher coefficients are found (Figure \ref{regress-beta} right).


\begin{figure}[!h]
    \centering
    \includegraphics[width=0.8\textwidth]{regression_residuals.png}
    \caption{{\bf Predicting age with tractometry and SGL.} Left: The predicted age of each individual (on the abscissa) and true age (on the ordinate), from the test splits (i.e., when each subject's data was held out in fitting the model); an accurate prediction falls close to the $y=x$ line (dashed). The mean absolute error in this case is 3.6 years and, the coefficient of determination $R^2=0.3$. Right: Standardized residuals (on the abscissa) as a function of the true age (on the ordinate). Predictions are generally more accurate for younger individuals.
    }
    \label{fig:regress-results}
\end{figure}

\begin{figure}[!h]
    \centering
    \includegraphics[width=0.3\textwidth]{regression_beta_annotated.pdf}
    \includegraphics[width=0.67\textwidth]{regression_tract_profiles.pdf}
    \caption{{\bf Feature importance for predicting age from tractometry.} Left: A skeletonized display of the main brain tracts analyzed, with anterior facing up, and right hemisphere on the right. The $\hat{\beta}$ coefficients displayed in blue (negative) to red (positive) are for measurements of FA along the tracts. The left cingulum cingulate (A) and forceps minor (B) are highlighted. Right: the FA (in shades of blue and green) and the $beta$ coefficients (dashed) in (A) left cingulum and (B) forceps minor.
    }
    \label{fig:regress-beta}
\end{figure}


\begin{itemize}
  \item Regression
    \begin{itemize}
      \item Lifespan maturation age regression
      \item Insert figure showing sparsity pattern
      \item Insert figure showing weights as they relate to tract differences visible in the browser
    \end{itemize}
\end{itemize}
\begin{itemize}
  \item Failures
    \begin{itemize}
      \item Hopefully, the failures are common to both regression
        and classification so we can include them here in there own
        subsection.
      \item Insert figure demonstrating failure cases
    \end{itemize}
\end{itemize}

\subsection*{Classification for ALS detection}

Using data from a previous study of the corticospinal tract profile and
ALS\cite{sarica2017corticospinal}, we tested the performance of SGL in a
classification setting. The previous study predicted ALS status with a mean
accuracy of 80\% using a random forest algorithm based on a priori selection of
features within the corticospinal tract. SGL delivers competitive predictive
performance (mean 93\% $\pm$ 2\% accuracy, 0.978 $\pm$ 0.006 ROC AUC) without
the need for a priori feature engineering. The results of the classification
prediction are shown in Fig~\ref{fig:class-results}, with ``ground-truth'' ALS
status separated into columns, and predicted ALS status encoded by color. In
addition to this classification performance, SGL also identifies the white
matter tracts most important for ALS classification. The relative importance of
white matter features is captured in the $\beta$ coefficients from
Eq~\eqref{eq:sgl}. Fig~\ref{fig:class-beta} depicts these coefficients across
the brain, laid out on a skeleton of the major tracts. We find that SGL selects
FA metrics in the corticospinal tract and particualrly in the right
corticospinal tract as most important to ALS classification, confirming previous
findings\cite{van2011upper, toosy2003diffusion, sarica2014tractography,
sage2007quantitative, sage2009quantitative, karlsborg2004corticospinal,
ellis1999diffusion, cosottini2005diffusion, ciccarelli2009investigation,
abe2010voxel} and identifying exactly the portions of the brain that were
selected \emph{a priori} in the previous study from which we collected our
data\cite{sarica2017corticospinal}.

\begin{figure}[!h]
    \centering
    \includegraphics[width=0.45\textwidth]{classification_probs.png}
    \includegraphics[width=0.45\textwidth]{classification_beta_bupu.png}

    \caption{{\bf Prediction of ALS status.}
        Left: Classification probabilities for each subject's ALS diagnosis.
        Controls are on the left while patients are on the right. Predicted
        controls are in blue and predicted patients are in red. Thus, false
        positive are represented as red dots on the left, while false negatives
        are represented as blue dots on the right. The SGL algorithm achieves
        93\% $\pm$ 2\% accuracy,with 0.978 $\pm$ 0.006 ROC AUC. Right: SGL
        coefficients are presented on a skeleton of the major tracts. The brain
        is oriented with the right hemisphere to our left and anterior out of
        the page. As expected large negative coefficients are in the FA of the
        corticospinal tract (and particularly right CST, here to the left)}
    \label{fig:class-results}
\end{figure}


\begin{figure}[!h]
    \centering
    \includegraphics[width=0.95\textwidth]{classification_tract_profiles.pdf}
    \caption{{\bf Coefficients with FA tract profiles}
       }
    \label{fig:class-profiles}
\end{figure}


\begin{figure}[!h]
    \centering
        \includegraphics[width=0.95\textwidth]{classification_subjects_profiles.pdf}
    \caption{{\bf Mis-classifications}
       They make sense}
    \label{fig:class-errors}
\end{figure}
