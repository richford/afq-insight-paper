\section*{Materials and methods}

\subsection*{Data}

Two different previously-published datasets were used here:

\begin{enumerate}

\item Diffusion MRI from a previous study of the corticospinal
tract (CST) in patients with amyotrophic lateral sclerosis
(ALS \cite{sarica2017corticospinal}), containing data from 24 ALS
patients and 24 demographically matched healthy controls. These data
were measured in a GE Discovery 750 3T MRI scanner at the Institute
of Bioimaging and Molecular Physiology in Catanzaro. Informed consent
was provided as approved by the Ethical Committee of the University
``Magna Graecia'' of Catanzaro. Voxel resolution was \num{2x2x2}
$\text{mm}^3$ and 27 non-colinear directions were measured with a
$b=1000$ $\frac{\text{sec}}{\text{mm}^2}$. Data was preprocessed to 
correct for subject motion and for eddy currents. The diffusion tensor
model \cite{basser1994mr} was fit in every voxel.
We will refer to this dataset as ALS.

\item Diffusion MRI data from a previous study of properties of
the white matter across the lifespan \cite{yeatman2014lifespan},
containing dMRI data from 76 subjects with ages 6-50. These data were
measured in a GE Discovery 750 3T MRI scanner at the Stanford Center
for Cognitive and Neurobiological Imaging. The Stanford University
IRB approved the procedures of this study. Informed consent was
obtained from each adult participant, and assent for participation
was provided by parents/guardians for children. Voxel resolution was
\num{2x2x2}$\text{mm}^3$ with 96 non-colinear directions measured with a
$b=2000$ $\frac{\text{sec}}{\text{mm}^2}$ and 30 non-colinear directions
measured with a $b=1000$ $\frac{\text{sec}}{\text{mm}^2}$. Data was
preprocessed to correct for subject motion and the diffusion tensor
model \cite{basser1994mr} was fit in every voxel, using a robust fit
\cite{chang2005restore}. These data were acquired using a dual spin echo
sequence, in which there is sufficient time for eddy currents to subside
between the application of the gradients and the image acquisition, so
no eddy current correction was applied. We will refer to this dataset
as WH.

\end{enumerate}

Data from both of these studies was processed in a similar manner,
using the Matlab Automated Fiber Quantification toolbox (AFQ)
\cite{yeatman2012tract}: streamlines representing fascicles of white
matter tracts were generated using a determinstic tractography algorithm
that follows the prinicpal diffusion direction of the diffusion tensor
in each voxel (STT) \cite{basser2000vivo}. Major tracts were identified
using multiple criteria: streamlines are selected as candidates for
inclusion in a bundle of streamlines that represents a tract if they
pass through known inclusion ROIs and do not pass through exclusion
ROIs \cite{wakana2007reproducibility}. In addition, a probabilistic
atlas is used to exclude streamlines which are unlikely to be part of
a tract \cite{Hua2008-sh}. Each streamline is resampled to 100 nodes
and the robust mean at each location is calculated by estimating the 3D
covariance of the location of each node and excluding streamlines that
are more than 5 standard deviations from the mean location in any node.
Finally, a bundle profile of tissue properties in each bundle was created
by interpolating the value of MRI maps of these tissue properties to the
location of the nodes of the resampled streamlines designated to each
bundle. In each of 100 nodes, the values are summed across streamlines,
weighting the contribution of each streamline by the inverse of the
mahalnobis distance of the node from the average of that node across
streamlines. This means that streamlines that are more representative of
the tract contribute more to the bundle profile, relative to streamlines
that are on the edge of the tract.

This process creates bundle profiles, in which diffusion measures
are quantified and averaged along twenty major fiber tracts. Here,
we use only the mean diffusivity (MD) and the fractional anisotropy
(FA) of the diffusion tensor, but additional dMRI-based maps or maps
based on other quantitative MRI measurements can also be used. These
bundle profiles, along with the phenotypical data we wish to explain
or predict, form the input to the SGL algorithm. In a domain-agnostic
machine learning context, the phenotypical data comprise the target
variables while the bundle profiles form the feature or predictor
variables (See Fig~\ref{fig:group-structure}).

\subsection*{Sparse Group Lasso}

Before fitting a model to the data, imputation and standardization are
performed. Missing node values (e.g., in cases where AFQ designates a node as
not-a-number) are imputed via linear interpolation. Care is taken not to
interpolate across the boundaries between different bundles. Some diffusion
metrics will have naturally larger variance than others and may therefore
dominate the objective function and make the SGL estimator unable to learn from
the lower variance metrics. For example, fractional anisotropy (FA) is bounded
between zero and one and could be overwhelmed by an unscaled higher-variance
metric like the mean diffusivity (MD). To prevent this we remove each feature's
mean and scale it to unit variance (z-score) using the
\lstinline{StandardScaler} from scikit-learn \cite{scikit-learn}. Scaling is
performed separately within each cross-validation set's training or testing data
to prevent leakage of information between the testing and training
sets\cite{kaufman2012leakage}.

After scaling and imputation, the tractometry data and target
phenotypical data can be organized in a linear model:
\begin{equation}
    y = \mathbf{X} \beta + \epsilon,
    \label{eq:lm}
\end{equation}
where $y$ is the phenotype -- categorical, such as a clinical diagnosis,
or continuous numerical, such as the subject's age. The tractometry
data is represented in the feature matrix $\mathbf{X}$, with rows
corresponding to different subjects, and columns corresponding
to diffusion measures at different nodes within each bundle. The
relationship between tractometric features and the phenotypic target is
characterized by the coefficients in $\beta$. The error term, $\epsilon$
is an unobserved random variable that captures the error in the model.
We denote our prediction of the targer phenotype as $\hat{y}$ and the
coefficients that produce this prediction as $\hat{\beta}$, so that
\begin{equation}
    \hat{y} = \mathbf{X} \hat{\beta},
    \label{eq:lm-approx}
\end{equation}
without the error term, $\epsilon$. In general, the feature matrix
$\mathbf{X}$ has dimensions $S \times (B \times N \times M)$, where $S$
is the number of subjects, $B$ is the number of white matter bundles,
$N$ is the number of nodes in each bundle, and $M$ is the number of
diffusion metrics calculated at each node. Typically, $B = 20$, $N =
100$, and $2 \le M \le 8$, resulting in $\sim 4,000 - 16,000$ features.
Conversely, many dMRI studies have between a few dozen and a few
hundred subjects, yielding a feature matrix that is wide and short.
Even in cases where more than a thousand subjects are measured (e.g.,
in the Human Connectome Project, where 1,200 subjects were measured
\cite{VanEssen2012}), the problem is ill-posed: the high dimensionality
of this data requires regularization to avoid overfitting and generate
interpretable results.

Here, we propose that in addition to regularizing the coefficients
in $\hat{\beta}$, we can also capitalize on our knowledge of the
group structure of the bundle profile features in $\mathbf{X}$. The
bundle-metric combinations form a natural grouping. For example, we
expect that MD features within the left arcuate fasciculus will
co-vary across individuals. Likewise for FA values within the right
corticospinal tract (CST) and so on. This group structure is represented
in Fig~\ref{fig:group-structure}, which depicts the linear model
$\hat{y} = \mathbf{X} \hat{\beta}$. Thus, we seek a regularization
approach that will fit a linear model with anatomically-grouped
covariates, where we expect to observe both groupwise sparsity, where
the number of groups (bundle/metric combinations) with at least one
non-zero coefficients is small, as well as within-group sparsity, where
the number of non-zero coefficients within each non-zero group is small.
The sparse group lasso (SGL) is a penalized regression technique that
satisfies exactly these criteria\cite{simon2013sparse}. It solves for a
coefficient vector
$\hat{\beta}$ that satisfies
\begin{equation}
    \hat{\beta} = \min_\beta \frac{1}{2}
    ||y - \displaystyle \sum_{\ell = 1}^{G}
    \mathbf{X}^{(\ell)} \beta^{(\ell)}||_2^2
    + \lambda_1 \displaystyle \sum_{\ell = 1}^{G}
    \sqrt{p_\ell} ||\beta^{(\ell)}||_2
    + \lambda_2 ||\beta||_1,
    \label{eq:sgl}
\end{equation}
where $G$ is the number of groups $\mathbf{X}^{(\ell)}$ is the submatrix
of $\mathbf{X}$ corresponding to group $\ell$, $\beta^{(\ell)}$ is
the coefficient vector for group $\ell$ and $p_\ell$ is the length of
$\beta^{(\ell)}$. In the tractomtetry setting, $G = T \times M$ and
$\forall \ell: p_\ell = 100$. The first term is the mean square error
loss, $L_{\text{mse}}$, as in the standard linear regression framework.
The second and third terms encourage groupwise sparsity and overall
sparsity, respectively. If $\lambda_1 = 0$ and $\lambda_2 = 1$, the
SGL reduces to the traditional lasso\cite{tibshirani1996regression}.
Conversely, if $\lambda_1 = 1$ and $\lambda_2 = 0$, the SGL reduces to
the group lasso\cite{yuan2006model}.

\begin{figure}[!h]
    \centering
    \includegraphics[width=0.65\textwidth]{dMRI_group_structure.png}
    \caption{{\bf dMRI group structure.}
        The phenotypical target data and tractometric features can
        be organized into a linear model, $\hat{y} = \mathbf{X}
        \hat{\beta}$. The feature matrix $\mathbf{X}$ is color-coded
        to reveal a natural group structure: the left (orange) group
        contains $k$ features from the inferior fronto-occipital
        fascicle (IFOF), the middle (green) group contains $k$ features
        from the corpus callosum, and the right (blue) group
        contains $k$ features from the uncinate. The coefficients in
        $\hat{\beta}$ follow the same natural grouping. Fascicle image
        reproduced with permission from Ref~\cite{yeatman2012tract}
        Figure 1.
    }
    \label{fig:group-structure}
\end{figure}


\subsubsection*{Incorporating transformations on $y$}

Often, the target variable $y$ is not in the domain in which the linear
model can be best fit to it. Equation \eqref{eq:lm-approx} can be slightly
modified as:
\begin{equation}
    \hat{y} = f^-1(\mathbf{X} \hat{\beta}),
    \label{eq:lm-transform}
\end{equation}
where the transformation function $f^{-1}$ characterizes the transform
applied to the data before fitting the linear coefficients. For example,
an often-used transform is the logarithmic transform:
\begin{equation}
    f(\hat{y}) = \log_n(\hat{y})
    \label{eq:log-nonlinearity}
\end{equation}
In this case, the model is parametrized by one additional fit parameter,
$n$.

\subsubsection*{Classification of categorical $y$}
When the phenotypical target variable is categorical, as in the case of
explaining or predicting the presence of a clinical diagnosis, the SGL is
readily adapted to logistic regression, where the probability of a target
variable belonging to an arbitrary defined ``true'' class is the logistic
function of the result of the linear model,
\begin{equation}
    p(\hat{y} = 1) = \frac{1}{1 + \exp(\mathbf{X} \hat{\beta})},
    \label{eq:logit}
\end{equation}
or equivalently, the mean squared error loss function in Eq~\eqref{eq:sgl} is
replaced with the cross-entropy loss, which for binary classification is the
negative log likelihood of the SGL classifier giving the ``true'' label:
\begin{equation}
    L_{\text{mse}} \rightarrow L_{\log} =
    -\left(y \log(p) + (1 - y) \log(1 - p)\right).
    \label{eq:logloss}
\end{equation}

\subsection*{Implementation, cross-validation and metaparameter optimization}

For given values of $\lambda_1$ and $\lambda_2$, the cost function in Eq~\eqref{eq:sgl} can be optimized using proximal gradient descent methods
\cite{parikh2014proximal} here implemented as a custom proximal operator that is
then optimized using the C-OPT library\cite{copt}. This supplies an estimate of
the optimal $\hat{\beta}$ given a particular set of values for the
meta-parameters $\lambda_1$ and $\lambda_2$.

To objectively evaluate the model and guard against over-fitting,
we used a nested cross-validation scheme, depicted in
Fig~\ref{fig:cross-val} for the categorical classification case.
The subjects (i.e. rows of the feature matrix $\mathbf{X}$ in
Fig~\ref{fig:group-structure} and Eq~\eqref{eq:lm}) are shuffled and
then decomposed into $k$ batches, hereafter referred to as folds. For
the ALS dataset we used $k=10$ and for the WH dataset $k=5$. For each
unique fold, we hold that fold out as the test\textsubscript{outer} set
and let the remaining data comprise the train\textsubscript{outer} set,
with the subscript indicating the depth of the nested decomposition.
We further decompose each train\textsubscript{outer} set into three
folds, and again for each unique fold, we hold out that fold as the
test\textsubscript{inner} set and let the remaining data comprise the
train\textsubscript{inner} set. At level-1 of the decomposition, we fit
an SGL model using fixed regularization meta-parameters $\lambda_1$
and $\lambda_2$, training the model using train\textsubscript{inner}
and evaluating the fit on test\textsubscript{inner}. We find
the optimal values for $\lambda_1$ and $\lambda_2$ using
hyperoptimization, implemented using the hyperopt library's \verb|fmin|
function\cite{Bergstra_2015} with a tree of Parzen estimators search
algorithm\cite{bergstra2011algorithms}. For continuous numerical $y$,
\verb|fmin| searches for meta-parameter values that minimize the median
absolute error. This can also be done minimizing the root of the mean
squared error (RMSE) or to maximizing the coefficient of determination
($R^2$). For binary categorical $y$ \verb|fmin| seeks to maximize the
classification accuracy. This can also be done maximizing the area
under the receiver operating curve (ROC AUC) or the average precision.
Using hyperoptimization, we find optimal regularization parameters and
$\hat{\beta}$ for each train\textsubscript{outer} set and then use those
to predict values for data in test\textsubscript{outer}. Thus each
subject in the dataset has a predicted phenotype derived from a model
that never saw its particular subject's data.

The above procedure describes $k$-fold cross validation. In fact, we use
repeated $k$-fold cross validation on the outer level of the decomposition, so
that the input data is decomposed into $k$ folds, three times. Thus, each
subject has three predicted phenotypes. We then take the mean predicted value to
summarize the performance of the model. In the classification case, the
shuffling into folds is stratified such that each fold has a population that
preserves the percentage of each class found in the larger input data.

\begin{figure}[!h]
    \centering
    \includegraphics[width=0.55\textwidth]{nested-cross-validation.pdf}
    \caption{{\bf Nested $k$-fold cross-validation scheme.}
        We evaluate model quality using a nested $k$-fold cross
        validation scheme. At level-0, the input data is decomposed
        three times into $k$ shuffled groups and optimal hyperparameters
        are found for the level-0 training set. Optimization of these
        hyperparameters requires the use of the hyperopt library and
        many repeated evaluations of an SGL model over a search space
        of possible regularization parameters. These evaluations take
        place at level-1 of the decomposition, where the level-0 training
        set is further decomposed into three shuffled groups.
        For the ALS data, $k=10$. For the WH data, $k=5$.
    }
    \label{fig:cross-val}
\end{figure}

\subsection*{Software implementation}

The full software implementation of the SGL approach presented here is available
in a Python software package called AFQ-Insight, which is developed publicly in
\url{https://github.com/richford/afq-insight}. The version of the code used to
produce the results herein is also available in \ariel{Need to add Zenodo DOI}.
AFQ-Insight reads the target and feature data that has been processed by AFQ
from comma separated value (CSV) files conforming to the AFQ-Browser data
format\cite{yeatman2018browser} and represents them internally as
\lstinline{DataFrame} objects from the pandas Python
library\cite{mckinney2010data}. The software provides different options for
imputing missing data in the feature matrix. Missing interior nodes are imputed
using linear interpolation. For missing exterior nodes, the user may choose
between linear extrapolation and constant forward(back)-fill. Imputation uses
only values from adjacent nodes in the same white matter bundle in the same
subject so there is no danger of data leakage from other subjects. It uses the
scikit-learn\cite{scikit-learn} library to decompose input data into separate
test and train datasets, to scale each feature to have zero mean and
unit variance, and to evaluate each fit in the hyperparameter search using
appropriate classification and regression metrics such as accuracy, area
under the receiver operating curve (AUC ROC), and coefficient of determination
($R^2$). For each set of hyperparameters, we solve the SGL using a custom
proximal operator supplied to the C-OPT library\cite{copt}. Appropriate
hyperparameters are found using the hyperopt library\cite{Bergstra_2015}.
