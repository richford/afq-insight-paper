\documentclass{letter}
\usepackage[margin=1in]{geometry}

\signature{Adam Richie-Halford, PhD \\ \vspace{4em} Ariel Rokem, PhD}
\address{The eScience Institute \\ University of Washington \\ Seattle, WA}
\begin{document}

\begin{letter}{}
\opening{Dear Editors,}

We are pleased to submit our manuscript entitled ``Multidimensional analysis
and detection of informative features in human brain white matter'' to
\emph{PLOS Computational Biology}.


This work presents a novel approach to the analysis of brain connectivity in
diffusion-weighted MRI. Adopting a group-sparse linear approach, we model the
multi-dimensional data from these measurements, taking into account the
grouping of the measurement to the different anatomical connections between
brain regions. We demonstrate the accuracy of this approach in four different
datasets. Our results show that the model is highly accurate (exceeding the
accuracy of previous state-of-the-art results) in both classification of
disease states, as well as the prediction of continuous variables (i.e., the
biological age of participants). Importantly, the results show that the
method produces results that are not only accurate but also highly
interpretable, because of the linear modeling approach. We produced
high-quality open-source software implementing our approach that can easily
be used in the analysis of other datasets.

Neuroscience, like many other fields, is at the cusp of an exciting era of
Big Data. This is thanks in large part to concerted efforts to collect and
aggregate large datasets. The paradigmatic example of this kind of data
collection is the NIH-funded Human Connectome Project (HCP), which has
collected high-quality MRI data from more than 1,200 young healthy
individuals, coupled with behavioral and genetic information. Similarly, the
Healthy Brain Network study, which we analyze in the present paper, as well
as the Adolescent Brain Cognitive Development study (ABCD) will each collect
cross-sectional and longitudinal data from approximately 10,000 children
during their development, together with detailed information about their
cognitive and mental health development. Several dozen other projects are
currently collecting large datasets of diffusion-weighted MRI in specific
populations, and addressing specific brain disease states. As these datasets
are collected, the field is in dire need of analysis methods that are both
predictive and interpretable. In our manuscript, we demonstrate that our
method is applicable to very large datasets, including two different
openly-available datasets with more than 500 subjects each.

Importantly, the software and analysis methods that we provide generalize
well beyond neuroscience into other fields where group-sparse linear models
can be applied, including other fields of biology, but also other physical
and even social sciences. Therefore, we believe that this manuscript will be
of interest to a broad range of researchers.

We appreciate your consideration of our manuscript

\closing{Sincerely,}

\encl{Manuscript}
\encl{Supplemental materials}

\end{letter}
\end{document}