\documentclass[11pt,letterpaper]{letter} % Specify the font size (10pt, 11pt and 12pt) and paper size (letterpaper, a4paper, etc)

\usepackage{graphicx} % Required for including pictures
\usepackage{microtype} % Improves typography
%\usepackage{gfsdidot} % Use the GFS Didot font: http://www.tug.dk/FontCatalogue/gfsdidot/

\usepackage[T1]{fontenc} % Required for accented characters
\usepackage{url}
\usepackage{fontspec}

\setmainfont{Latin Modern Sans}

% Create a new command for the horizontal rule in the document which allows thickness specification
\makeatletter
\def\vhrulefill#1{\leavevmode\leaders\hrule\@height#1\hfill \kern\z@}
\makeatother

%----------------------------------------------------------------------------------------
%	DOCUMENT MARGINS
%----------------------------------------------------------------------------------------

\textwidth 7in
\textheight 10.0in
\oddsidemargin -.25in
\evensidemargin -.25in
\topmargin -1.in
\longindentation 0.50\textwidth
\parindent 0in

%----------------------------------------------------------------------------------------
%	SENDER INFORMATION
%----------------------------------------------------------------------------------------

\def\Who{Ariel Rokem} % Your name
\def\What{, PhD} % Your title
\def\Where{Department of Psychology} % Your department/institution
\def\Address{Guthrie Hall 119  A} % Your address
\def\CityZip{Seattle, WA 98105} % Your city, zip code, country, etc
\def\Email{E-mail: arokem@uw.edu} % Your email address
\def\TEL{Phone: +1 (510) 387-6264} % Your phone number
\def\URL{http://arokem.org} % Your URL

%----------------------------------------------------------------------------------------
%	HEADER AND FROM ADDRESS STRUCTURE
%----------------------------------------------------------------------------------------

\address{
\begin{flushleft}
\includegraphics[width=1in]{UWLogo.png} % Include the logo of your institution
\end{flushleft}
\hspace{5in} % Position of the institution logo, increase to move left, decrease to move right
\vskip -1.7in~\\ % Position of the text in relation to the institution logo, increase to move down, decrease to move up
\Large\hspace{1.5in} \hfill ~\\[0.05in] % First line of institution name, adjust hspace if your logo is wide
\hspace{1.5in} \hfill \normalsize % Second line of institution name, adjust hspace if your logo is wide
%\makebox[0ex][r]{\bf \Who \What }\hspace{0.08in} % Print your name and title with a little whitespace to the right
~\\[-0.11in] % Reduce the whitespace above the horizontal rule
\hspace{1.5in}%\vhrulefill{1pt} \\ % Horizontal rule, adjust hspace if your logo is wide and \vhrulefill for the thickness of the rule
\hspace{\fill}\parbox[t]{2.95in}{ % Create a box for your details underneath the horizontal rule on the right
\footnotesize % Use a smaller font size for the details
\Who \What \\ \em % Your name, all text after this will be italicized
\Where\\ % Your department
\Address\\ % Your address
\CityZip\\ % Your city and zip code
\TEL\\ % Your phone number
\Email\\ % Your email address
\URL % Your URL
}
\hspace{-1.4in} % Horizontal position of this block, increase to move left, decrease to move right
\vspace{-1in} % Move the letter content up for a more compact look
}

%----------------------------------------------------------------------------------------
%	TO ADDRESS STRUCTURE
%----------------------------------------------------------------------------------------

\def\opening#1{\thispagestyle{empty}
{\centering\fromaddress \vspace{1.2in} \\ % Print the header and from address here, add whitespace to move date down
\hspace*{\longindentation}\today\par} % Print today's date, remove \today to not display it
{\raggedright \toname \\ \toaddress \par} % Print the to name and address
\vspace{0.2in} % White space after the to address
\noindent #1 % Print the opening line
% Uncomment the 4 lines below to print a footnote with custom text
%\def\thefootnote{}
%\def\footnoterule{\hrule}
%\footnotetext{\hspace*{\fill}{\footnotesize\em Footnote text}}
%\def\thefootnote{\arabic{footnote}}
}

%----------------------------------------------------------------------------------------
%	SIGNATURE STRUCTURE
%----------------------------------------------------------------------------------------

\signature{\Who \What} % The signature is a combination of your name and title

\long\def\closing#1{
\vspace{0.1in} % Some whitespace after the letter content and before the signature
\noindent % Stop paragraph indentation
\hspace*{\longindentation} % Move the signature right
\parbox{\indentedwidth}{\raggedright
#1 % Print the signature text
\vskip 0.65in % Whitespace between the signature text and your name
\fromsig}} % Print your name and title

%----------------------------------------------------------------------------------------

\begin{document}

\begin{letter}{}
\opening{Dear Editors,}

We are pleased to submit our manuscript entitled ``Multidimensional analysis
and detection of informative features in human brain white matter'' to
\emph{PLOS Computational Biology}.

This work presents a novel approach to the analysis of brain connectivity in
diffusion-weighted MRI. Adopting a group-sparse linear approach, we model the
multi-dimensional data from these measurements, taking into account the
grouping of the measurement to the different anatomical connections between
brain regions. We demonstrate the value of this approach in four different
datasets. Our results show that the model is highly accurate (exceeding the
accuracy of previous state-of-the-art results) in both classification of
disease states, as well as the prediction of continuous variables (i.e., the
biological age of participants). Importantly, the results show that the
method produces results that are not only accurate but also highly
interpretable, because of the linear modeling approach. We produced
high-quality open-source software implementing our approach that can easily
be used in the analysis of other datasets (available at \url{https://github.com/richford/AFQ-Insight}).

Neuroscience, like many other fields, is at the cusp of an exciting era of
Big Data. This is thanks in large part to concerted efforts to collect and
aggregate large datasets. The paradigmatic example of this kind of data
collection is the NIH-funded Human Connectome Project (HCP), which has
collected high-quality MRI data from more than 1,200 young healthy
individuals, coupled with behavioral and genetic information. Similarly, the
Healthy Brain Network study, which we analyze in the present paper, as well
as the Adolescent Brain Cognitive Development study (ABCD) will each collect
cross-sectional and longitudinal data from approximately 10,000 children
during their development, together with detailed information about their
cognitive and mental health development. Several dozen other projects are
currently collecting large datasets of diffusion-weighted MRI in specific
populations, and addressing specific brain disease states. As these datasets
are collected, the field is in dire need of analysis methods that are both
predictive and interpretable. In our manuscript, we demonstrate that our
method is applicable to very large datasets, including two different
openly-available datasets with more than 500 subjects each.

Importantly, while the focus of our paper is analysis of human brain white
matter, and this will translate immediately to work by other neuroscientists,
the software and analysis methods that we provide generalize well beyond
neuroscience into other fields where group-sparse linear models can be applied,
including other fields of biology, such as genetics and ecology. Therefore, we
believe that this manuscript will be of interest to a broad range of
researchers.

We appreciate your consideration of our manuscript

\hspace{1cm} Sincerely,\\

\begin{minipage}{0.8\linewidth}
\hspace{1cm}\begin{tabular}{ll}
Ariel Rokem, PhD & \hspace{1cm} Adam Richie-Halford, PhD\\
Research Assistant Professor & \hspace{1cm}  Postdoctoral Fellow \\
Department of Psychology & \hspace{1cm} eScience Institute \\
University of Washington & \hspace{1cm} University of Washington
 \end{tabular}
\end{minipage}


\end{letter}
\end{document}
