%% ****** Start of file apstemplate.tex ****** %
%%
%%   This file is part of the APS files in the REVTeX 4 distribution.
%%   Version 4.1r of REVTeX, August 2010
%%
%%
%%   Copyright (c) 2001, 2009, 2010 The American Physical Society.
%%
%%   See the REVTeX 4 README file for restrictions and more information.
%%
%
% This is a template for producing manuscripts for use with REVTEX 4.0
% Copy this file to another name and then work on that file.
% That way, you always have this original template file to use.
%
% Group addresses by affiliation; use superscriptaddress for long
% author lists, or if there are many overlapping affiliations.
% For Phys. Rev. appearance, change preprint to twocolumn.
% Choose pra, prb, prc, prd, pre, prl, prstab, prstper, or rmp for journal
%  Add 'draft' option to mark overfull boxes with black boxes
%  Add 'showpacs' option to make PACS codes appear
%  Add 'showkeys' option to make keywords appear
\documentclass[10pt,%
               aps,%
               prl,%
               preprint,%
               superscriptaddress,%
               preprintnumbers,%
               amsmath,%
               floatfix,%
               endfloats*]{revtex4-2}
\usepackage[T1]{fontenc}
\usepackage[utf8x]{inputenc} 
\usepackage[export]{adjustbox}  % Used to adjust figure frames
\usepackage{subcaption}
\usepackage{url}
\usepackage{color}
\usepackage{xcolor}
\usepackage{listings}
\usepackage{float}
\usepackage{amsfonts}
\usepackage{mathtools}
\usepackage{dcolumn}
\usepackage{wrapfig}
\usepackage{physics}
\usepackage[inline]{enumitem}
\usepackage[per-mode=symbol,group-separator={,},group-minimum-digits=4]{siunitx}
\usepackage{todonotes}
\usepackage{graphicx}
\graphicspath{ {../paper_figures} }

%%% HELPER CODE FOR DEALING WITH EXTERNAL REFERENCES
\usepackage{xr-hyper}
\usepackage[pdftex,%
            colorlinks=true,%
            linkcolor=blue,%
            citecolor=blue,%
            urlcolor=blue,%
            bookmarksnumbered=true,%
            bookmarksopen=true]{hyperref}
\usepackage[capitalize]{cleveref}
\makeatletter
\newcommand*{\addFileDependency}[1]{
  \typeout{(#1)}
  \@addtofilelist{#1}
  \IfFileExists{#1}{}{\typeout{No file #1.}}
}
\makeatother

\newcommand*{\myexternaldocument}[2][ext:]{
    \externaldocument[#1]{#2}
    \addFileDependency{#2.tex}
    \addFileDependency{#2.aux}
}
%%% END HELPER CODE

% put all the external documents here!
\myexternaldocument[supp:]{supplemental}

\begin{document}

\title{Multidimensional analysis and detection of informative features in human white matter}

\author{Adam \surname{Richie-Halford}}%
\email{richford@uw.edu}%
\affiliation{%
    eScience Institute,%
    University of Washington, Seattle,%
    Washington 98195--1560, USA
}
  
\author{Jason \surname{Yeatman}}%
\affiliation{%
    Graduate School of Education and Division of Developmental and Behavioral Pediatrics,%
    Stanford University,%
    Stanford, CA, 94305, USA
}

\author{Noah \surname{Simon}}%
\affiliation{%
    Department of Biostatistics,%
    University of Washington, Seattle,%
    Washington 98195--1560, USA
}

\author{Ariel \surname{Rokem}}%
\affiliation{%
    Department of Psychology,%
    University of Washington, Seattle,%
    Washington 98195--1560, USA
}
  
\date{\today}

% Please keep the abstract below 300 words
\newcommand*{\alsLRatio}{$0.21$}
\newcommand*{\whLRatio}{$0.84$}
\newcommand*{\hbnLRatio}{$0.72$}
\newcommand*{\ccLRatio}{$0.83$}
\newcommand*{\alsAccuracy}{$83$}
\newcommand*{\alsRocAuc}{$0.88$}
\newcommand*{\whRsq}{$0.52$}
\newcommand*{\whMae}{$2.84$}
\newcommand*{\hbnRsq}{$0.56$}
\newcommand*{\hbnMae}{$1.5$}
\newcommand*{\camcanRsq}{$0.77$}
\newcommand*{\camcanMae}{$6.02$}

\section*{Abstract}

The white matter contains long-range connections between different
brain regions and the organization of these connections holds important
implications for brain function in health and disease. Tractometry
uses diffusion-weighted magnetic resonance imaging (dMRI) data
to quantify tissue properties (e.g. fractional anisotropy (FA),
mean diffusivity (MD), etc.), along the trajectories of these
connections\cite{yeatman2012tract}. Statistical inference from
tractometry usually either (a) averages these quantities along
the length of each bundle in each individual, or (b) performs analysis
point-by-point along each bundle, with group comparisons or regression
models computed separately for each point along every one of the bundles.
These approaches are limited in their sensitivity, in the former case, or
in their statistical power, in the latter.
In the present work, we developed a method based on the sparse group
lasso (SGL) \cite{simon2013sparse} that takes into account tissue
properties measured along all of the bundles, and selects informative
features by enforcing sparsity, not only at the level of individual
bundles, but also across the entire set of bundles and all of the measured
tissue properties. The sparsity penalties for each of these constraints
is identified using a nested cross-validation scheme that guards
against over-fitting and simultaneously identifies the correct
level of sparsity. We demonstrate the accuracy of the method in two
settings: i) In a classification setting, patients with amyotrophic
lateral sclerosis (ALS) are accurately distinguished from matched
controls \cite{sarica2017corticospinal}. Furthermore, SGL automatically
identifies FA in the corticospinal tract as important for this
classification -- correctly finding the parts of the white matter known
to be affected by the disease. ii) In a regression setting, dMRI is
used to accurately predict ``brain age'' \cite{yeatman2014lifespan,
Brown2012-so}. In this case, the weights are distributed throughout the
white matter indicating that many different regions of the white matter
change with development and contribute to the prediction of age. Thus,
SGL makes it possible to leverage the multivariate relationship between
diffusion properties measured along multiple bundles to make accurate
predictions of subject characteristics while simultaneously discovering
the most relevant features of the white matter for the characteristic of
interest.


\maketitle

\section*{Introduction}

Diffusion-weighted Magnetic Resonance Imaging (dMRI) provides a unique view into
the physical properties of the connections that comprise the brain white matter.
While the measurements are usually conducted with voxels at the millimeter
scale, water molecules within each voxel probe the structure of white matter at
the micrometer scale providing important information about the physical
structure of the white matter, including the density of axons and distribution
of nerve fascicle orientations within each voxel \cite{wandell2016clarifying}.
Even though metrics derived from diffusion measurements are ambiguous in terms
of their underlying biological interpretation \cite{Jones2013-xv}, analyzing the
variance in these properties has proven useful in characterizing individual
differences in cognitive function, characterizing differences between
populations and detecting brain abnormalities associated with disease
\cite{Thomason2011-qn}.

To relate the diffusion in each voxel to the macro-structure of long-range
connections between different brain regions, methods for computational
tract-tracing from diffusion MRI, or tractography, combine the estimates of
fascicle orientations in each voxel to form streamlines that traverse the volume
of the white matter \cite{Conturo1999-je, Mori2002-qi}. These methods have been
under increased scrutiny and several lines of investigation have raised
critiques of their validity \cite{Maier-Hein2017-vb, Thomas2014-ki}. On the
other hand, there have been efforts to shore up the inferences made with these
methods \cite{Pestilli2014NatMeth, Takemura2016-sh, Smith2013-nc, Smith2015-cx,
Smith2015-zt, Rheault2018-wk}. Importantly, though discovering novel nerve fiber
bundles requires extraordinary evidence, and delineating the exact cortical
termination of the streamlines in the gray matter is still prone to error, there
is little dispute that tractography can accurately define the location of
several major white matter bundles that are known to exist within the core of
the white matter \cite{Maier-Hein2017-vb}.

Leveraging this fact, one of the most powerful methods currently
available to put macro- and micro-structure together is
\emph{tractometry}: assessment of the physical properties of the white
matter along specific tracts \cite{Bells2011-cf}. Though there are
several different available implementations of this overall idea,
the principles are similar \cite{yeatman2012tract, Yendiki2011-ay,
Wassermann2016-iv, ODonnell2009-uu}: tractometry begins by delineating
the parts of the white matter that belong to different major tracts or
axonal nerve fiber bundles, such as the corticospinal tract or arcuate
fasciculus, and samples the biophysical properties (such as fractional
anisotropy or mean diffusivity) along the length of these bundles.
In some cases, tissue properties along the length of each bundle are
summarized by taking the mean along each bundle, but there is a large
body of evidence showing that there is systematic variability along
the trajectory of each bundle. This justifies retaining the individual
samples along the length of each bundle \cite{yeatman2012tract,
colby2012, ODonnell2009-uu}. However, while this retains important
information about each individual's white matter, it also presents
statistical challenges due to the dimensionality of the data. Based on
tractometry, researchers may choose to compare different individuals
to each other. This is usually done following one of the following
approaches:

\begin{enumerate}

\item Mass univariate approaches: In this approach comparisons between
groups or across individuals are done independently at each node
of each tract, for each one of the diffusion metrics available at
that point. This approach is exhaustive, but statistical power is
compromised by a multiple comparison problem. Different approaches
can be taken to resolving this challenge. For example, Colby and
colleagues \cite{colby2012} used a non-parametric resmapling approach to
correct for family-wise error across the different possible comparisons
\cite{Nichols2002-zu, Nichols2003-yy}.

\item Region of interest(ROI)-based approaches: An alternative that
circumvents the multiple comparison problem is to select just a few
tracts to compare in each individual, or even focusing on particular
segments of these tracts based on \emph{a priori} hypotheses. This
approach is very powerful when the biological basis of the process
of interest is relatively well understood (for a recent example, see
\cite{huber2018rapid}).

\item ROI-based selection, followed by multivariate analysis: Here,
an ROI is selected based on \emph{a priori} knowledge, and all the
nodes or voxels in the ROI are used together to fit a model that can
predict differences between individuals. An example of that is the
``profilometry'' framework, in which different diffusion metrics
from a single tract are combined together to provide input to a
multivariate analysis of covariance, and linear discriminant analysis
\cite{dayan2016profilometry}.

\end{enumerate}

Generally speaking, analysis methods should balance predictive
accuracy with descriptive power \cite{Murdoch2019-ax, Breiman2001-uz}.
Accordingly, tractometry analysis should simultaneously capitalize on
all the data across all tracts to make the best possible prediction,
while also retaining and elucidating spatial information about the
locations that are most informative for a prediction. In the present
work, we developed a novel framework for analysis of tractometry that
simultaneously selects the features for analysis, and fits a model
to these features. We use a linear modeling approach, which aims to
predict phenotypical variance in a group of subjects, based on a linear
combination of the features estimated with tractometry.

Using this approach, we first need to deal with the large and
asymmetric dimensionality of the data: tractometry data usually has
many more features (i.e., number of measurements per individual) than
samples (number of subjects), which makes inferences from the data
about phenotypical differences between individuals ill-posed. This
regime is the target of several statistical learning techniques, and
is often solved by various forms of regularization. For example,
Tikhonov regularization shrinks the solution such that the sum of
squared contributions from the individual features are minimized
\cite{Hoerl2000-ij}. Another solution to the problem is provided by
the Lasso algorithm, which instead minimizes the sum of the absolute
values of contributions of each feature \cite{Tibshirani1996-qs}. This
tends to shrink to zero the contributions of many of the features,
providing results that are both accurate and interpretable. When
additional structure is available in the organization of the data,
regularization algorithms can take advantage of this structure. For
example, if the features lend themselves to a natural division into
different groups, the group lasso (GL) can be used to select groups
of features, rather than individual features \cite{Yuan2006-ky}.
The Sparse Group Lasso (SGL) elaborates on this idea by providing
control both of group sparsity, as well as overall sparsity of the
solutions \cite{simon2013sgl}. Because the features measured with
tractomery lend themselves to grouping based on the tracts from which
each measurement is taken, GL and SGL could provide a useful tool
for linear model fitting in problems of this form. Here we, first,
develop an implementation of SGL that is well suited to the analysis of
tractometry data and, second, demonstrate the power and flexibility of
this approach by applying it to both classification (disease diagnosis)
and continuous prediction (age) problems from previously published
studies \cite{sarica2017corticospinal, yeatman2014lifespan}.

\section*{Results and discussion}

Using data from a previous study of the corticospinal tract profile and ALS\cite{sarica2017corticospinal}, we tested the performance of AFQ-Insight in a classification setting. The previous study predicted ALS status with a mean accuracy of 80\% using a random forest algorithm based on a priori selection of features within the corticospinal tract. AFQ-Insight delivers competitive predictive performance (mean 84\% accuracy, 0.93 ROC AUC) without the need for a priori feature engineering. The results of the classification prediction are shown in Fig~\ref{fig:class-results}, with ``ground-truth'' ALS status separated into columns, and predicted ALS status encoded by color. In addition to this classification performance, AFQ Insight also identifies the white matter tracts most important for the ALS classification. The relative importance of white matter features is captured in the $\beta$ coefficients from Eq~\eqref{eq:sgl}. Fig~\ref{fig:class-beta} depicts these coefficients ``unfolded'' across the brain. The $x$-axis organizes white matter tracts starting with left distal tracts, progressing toward the center of the brain and then outward toward the right distal tracts. Fig~\ref{fig:class-beta} shows that AFQ-Insight selects FA metrics in the right corticospinal tract as most important to ALS classification, confirming previous findings\cite{van2011upper, toosy2003diffusion, sarica2014tractography, sage2007quantitative, sage2009quantitative, karlsborg2004corticospinal, ellis1999diffusion, cosottini2005diffusion, ciccarelli2009investigation, abe2010voxel} and identifying exactly the portions of the brain that were selected \emph{a priori} in the previous study from which we collected our data\cite{sarica2017corticospinal}.

\begin{figure}[!h]
    \centering
    \includegraphics[width=0.65\textwidth]{classification_results.png}
    \caption{{\bf Prediction of ALS status.}
        Classification probabilities for each subject's ALS diagnosis. Controls are on the left while patients are on the right. Predicted controls are in blue and predicted patients are in red. Thus, false positive are represented as red dots on the left, while false negatives are represented as blue dots on the right. AFQ-Insight achieves 84\% accuracy with an ROC area under the curve of 0.93.
    }
    \label{fig:class-results}
\end{figure}

\begin{figure}[!h]
    \centering
    \includegraphics[width=0.95\textwidth]{classification_unfolded_beta.png}
    \caption{{\bf Feature importance for prediction of ALS status.}
        Classification probabilities for each subject's ALS diagnosis. Controls are on the left while patients are on the right. Predicted controls are in blue and predicted patients are in red. Thus, false positive are represented as red dots on the left, while false negatives are represented as blue dots on the right. AFQ-Insight achieves 84\% accuracy with an ROC area under the curve of 0.93.
    }
    \label{fig:class-beta}
\end{figure}

\begin{itemize}
  \item Classification
    \begin{itemize}
      \item Insert figure showing weights as they relate to tract differences visible in the browser
    \end{itemize}
\end{itemize}

In a regression setting, we attempted to accurately predict ``brain age''.

\begin{figure}[!h]
    \centering
    \includegraphics[width=0.65\textwidth]{regression_results.png}
    \caption{{\bf Prediction of brain age.}
    }
    \label{fig:class-results}
\end{figure}

\begin{figure}[!h]
    \centering
    \includegraphics[width=0.95\textwidth]{regression_unfolded_beta.png}
    \caption{{\bf Feature importance for brain age.}
    }
    \label{fig:class-beta}
\end{figure}

\begin{itemize}
  \item Regression
    \begin{itemize}
      \item Lifespan maturation age regression
      \item Insert figure showing sparsity pattern
      \item Insert figure showing weights as they relate to tract differences visible in the browser
    \end{itemize}
\end{itemize}
\begin{itemize}
  \item Failures
    \begin{itemize}
      \item Hopefully, the failures are common to both regression
        and classification so we can include them here in there own
        subsection.
      \item Insert figure demonstrating failure cases
    \end{itemize}
\end{itemize}

\section*{Conclusion}

We present here a novel method for analysis of dMRI tractometry data. This
method relies on the Sparse Group Lasso \cite{simon2013sparse} to provide both
acurate prediction of the phenotypic properties of individual subjects based on
their dMRI data, but also provides interpretable results by identifying the
features that are important across subjects to make these predictions. The
method is broadly applicable: it performs well in predicting both continuous
variables, such as biological age, as well as discrete variable, such as whether
a person is a patient or a healthy control. In both of these cases, SGL
out-performs previous algorithms that have been developed for these tasks. In
addition, the nested cross-validation approach that we have developed tunes the
degree of sparseness required by the algorithm, such that both very local
phenomena, as well as widely distributed phenomena are accurately captured.

Specifically, we demonstrated that the algorithm correctly detects the fact that
 ALS, which is a disease of lower motor neurons, is localized to the
 cortico-spinal tract. This recapitulates the results of previous analysis of
 these same data, using a targetted ROI-based approach. In contrast, in analysis
 of biological age, the coefficients identified by the algorithm are very widely
 distributed.

Taken together, these results demonstrate the promise of a group regularized
regression approach. Even at the scale of dozens of subjects, the results
provided by SGL are both accurate and interpretable. We expect that additional
data will improve the performance of these algorithms. Neuroscience has entered
an era in which large consortium efforts are putting together large datasets
that can be analyzed using this approach. Future work will apply this method to
these datasets. For example, to the Human Connectome Project dataset
\cite{VanEssen2012}

These results also motivate extensions of the method using more sophisticated
cost functions. For example, the fused sparse group lasso (FSGL)
\cite{zhou2012} extends SGL to enforce additional spatial structure: smoothness
in the variation of diffusion metrics along the tracts. As brain measurements
include additional structure (for example, bilateral symmetry), future work
could also incorporate overlapping group membership for each entry in the tract
profiles. For example, a measurement could come from the corpus callosum, but
also from the right hemipshere. This would also require extending the cost
function used here to incorporate these constraints.

The method is packaged as open-source software called AFQ-Insight that is openly
available. The sofware integrates within a broader automated fiber
quantification software ecosystem: AFQ \cite{yeatman2012tract}, which extracts
tractometry data from raw dMRI, as well as AFQ-Browser, which visualizes
tractometry data and facilitates sharing of the results of dMRI studies
\cite{yeatman2018browser}.

\begin{itemize}
  \item Feature selection + prediction
  \item Part of the AFQ ecosystem
  \item Open source, reproducibility
\end{itemize}

\section{Methods}

\begin{itemize}
  \item Data
    \begin{itemize}
      \item Raw measurement sources
      \item AFQ pipeline
      \item Insight preprocessing
    \end{itemize}
  \item Sparse Group Lasso
    \begin{itemize}
      \item Regression case
      \item Classification with logistic regression
    \end{itemize}
  \item Computational implementation
    \begin{itemize}
      \item Insert Figure for pipeline
      \item Proximal gradient methods
      \item Meta-parameter optimization
      \item Cross-validation scheme
        \begin{itemize}
          \item Insert Figure for cross-validation scheme
        \end{itemize}
    \end{itemize}
\end{itemize}


\begin{acknowledgments}
    {\it Acknowledgments.-}
    This work was supported by BRAIN Iniative grant 1RF1MH121868-01 from the
    National Institutes for Mental Health and by a grant from the Gordon \& Betty
    Moore Foundation and the Alfred P. Sloan Foundation to the University of
    Washington eScience Institute Data Science Environment. We would like to thank
    Scott Murray for a useful discussion of the SGL method and Mareike Grotheer for
    helpful comments on the manuscript.
\end{acknowledgments}

\printfigures

\bibliographystyle{naturemag}
\bibliography{paper}

\end{document}
